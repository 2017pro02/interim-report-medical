% プロジェクト学習中間報告書書式テンプレート ver.1.0 (iso-2022-jp)

% 両面印刷する場合は `openany' を削除する
\documentclass[openany,11pt,papersize]{jsbook}

% 報告書提出用スタイルファイル
%\usepackage[final]{funpro}%最終報告書
\usepackage[middle]{funpro}%中間報告書

% 画像ファイル (EPS, EPDF, PNG) を読み込むために
\usepackage[dvipdfmx]{graphicx,color,hyperref}

% ここから -->
\usepackage{calc,ifthen}
\newcounter{hoge}
\newcommand{\fake}[1]{\whiledo{\thehoge<70}{#1\stepcounter{hoge}}%
  \setcounter{hoge}{0}}
% <-- ここまで 削除してもよい

% ファイル分割のためのパッケージ
\usepackage{subfiles}

% url表記
\usepackage{url}

% 年度の指定
\thisYear{2017}

% プロジェクト名
\jProjectName{使ってもらって学ぶフィールド指向システムデザイン2017}

% [簡易版のプロジェクト名]{正式なプロジェクト名}
% 欧文のプロジェクト名が極端に長い(2行を超える)場合は、短い記述を
% 任意引数として渡す。
%\eProjectName[Making Delicious curry]{How to make delicious curry of Hakodate}
\eProjectName{Field Oriented System Design Learning by User’s Feedback}


% <プロジェクト番号>-<グループ名>
\ProjectNumber{2-C}

% グループ名
\jGroupName{医療グループ}
\eGroupName{Medical Group}

% プロジェクトリーダ
\ProjectLeader{1015061}{西谷歩}{Ayumi~Nishiya}

% グループリーダ
\GroupLeader  {1015174}{山根春貴}{Haruki~Yamane}

% メンバー数
\SumOfMembers{4}
% グループメンバ
\GroupMember  {1}{1015117}{佐藤碧}{Aoi~Sato}
\GroupMember  {2}{1015078}{佐藤礼於}{Leo~Sato}
\GroupMember  {3}{1015216}{堀沙枝香}{Saeka~Hori}
\GroupMember  {4}{1015259}{頼亜弥}{Aya~Rai}

% 指導教員
\jadvisor{伊藤恵,南部美砂子,奥野拓}
% 複数人数いる場合はカンマ(,)で区切る。カンマの前後に空白は入れない。
\eadvisor{Kei~Ito,Misako~Nambu,Taku~Okuno}

% 論文提出日
\jdate{2017年7月26日}
\edate{July~26, 2017}

\begin{document}
%
% 表紙
\maketitle

%前付け
\frontmatter

% 概要
\subfile{sections/0-overview}

\tableofcontents% 目次


\mainmatter% 本文のはじまり

\chapter{背景と目的}
\subfile{sections/1-haikei}


\chapter{活動プロセス}
\subfile{sections/2-katsudou}


\chapter{知識習得}
\subfile{sections/3-tchishiki}


\chapter{システム案の変化}
\subfile{sections/4-henka}


\chapter{中間成果発表}
\subfile{sections/5-tyukan}


\chapter{学びと今後の展望}
\subfile{sections/6-kongo}


% 以降、付録(付属資料)であることを示す
\begin{appendix}

\chapter{その他新規習得技術}
\subfile{sections/z1-gizyutsu}

\chapter{活用した講義}
\subfile{sections/z2-kougi}

%付録の終わり
\end{appendix}


%\backmatter

% 参考文献
\begin{thebibliography}{9}
    \bibitem{ueki} アルツハイマー型痴呆と栄養 (特集 アルツハイマー型痴呆のリスクファクター). 植木 彰, 2005. \\ \url{http://ci.nii.ac.jp/naid/40006750362/} (2016/07/15 アクセス)
    \bibitem{syourai} 日本における認知症の高齢者人口の将来推計に関する研究. 二宮 利治, 2014. \\ \url{https://mhlw-grants.niph.go.jp/niph/search/NIDD00.do?resrchNum=201405037A} (2017/7/17 アクセス)
    \bibitem{seikatsu} 羽生 春夫. 生活習慣病と認知症. 日老医誌, 2013.
    \bibitem{kaigo} 社会保障審議会福祉部会. 2025 年に向けた介護人材の確保 ~量と質の好循環の確立に向けて~. 福祉人材確保専門委員会, 2015.
    \bibitem{lifelog} 佐藤生馬他. 認知症高齢者向けライフログを用いた傾聴支援システムの有効性の検証. 生活生命支援医療福祉工学系学会連合大会, 2014.
    \bibitem{haikai} 寺尾北地区社会福祉協議会. 寺尾北地区社会福祉協議会だより. 広報委員会, 2014.
    \bibitem{} 飯干 紀代. 『今日から実践 認知症の人とのコミュニケーション 感情と行動を理解するためのアプローチ』. 中央法規, 2011.
    \bibitem{} 井庭崇・岡田誠. 『旅のことば』. 丸善出2015.
    \bibitem{} 永田久美子・桑野康一・諏訪免典子. 『認知症の人の見守り・SOS ネットワーク実例集 安心・安全に暮らせるまちを目指して』. 中央法2011.
    \bibitem{} 中島京子. 『長いお別れ』, ハヤカワ・ミステリ文庫, 1976.
    \bibitem{} 酒井保治郎・小宮桂治. 『よくわかる脳の障害とケア』. 南江堂, 2013.
    \bibitem{} 鈴木正典. 『認知症のための回想法』. 日本看護協会出版会, 2013.
    \bibitem{} 高橋龍太郎. 『楽しくいきいき、認知症予防!』. インターメディカ, 2013.
    \bibitem{} 植田孝一郎・鈴木明子・大塚俊男. 『認知症の人のための作業療法の手引き. ワールドプランニング』, 2010.
\end{thebibliography}


\end{document}
