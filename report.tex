% プロジェクト学習中間報告書書式テンプレート ver.1.0 (iso-2022-jp)

% 両面印刷する場合は `openany' を削除する
\documentclass[openany,11pt,papersize]{jsbook}

% 報告書提出用スタイルファイル
%\usepackage[final]{funpro}%最終報告書
\usepackage[middle]{funpro}%中間報告書

% 画像ファイル (EPS, EPDF, PNG) を読み込むために
\usepackage[dvipdfmx]{graphicx,color}

% ここから -->
\usepackage{calc,ifthen}
\newcounter{hoge}
\newcommand{\fake}[1]{\whiledo{\thehoge<70}{#1\stepcounter{hoge}}%
  \setcounter{hoge}{0}}
% <-- ここまで 削除してもよい

% ファイル分割のためのパッケージ
\usepackage{subfiles}

% 年度の指定
\thisYear{2017}

% プロジェクト名
\jProjectName{使ってもらって学ぶフィールド指向システムデザイン2017}

% [簡易版のプロジェクト名]{正式なプロジェクト名}
% 欧文のプロジェクト名が極端に長い(2行を超える)場合は、短い記述を
% 任意引数として渡す。
%\eProjectName[Making Delicious curry]{How to make delicious curry of Hakodate}
\eProjectName{Field Oriented System Design Learning by User's Feedback}


% <プロジェクト番号>-<グループ名>
\ProjectNumber{2-C}

% グループ名
\jGroupName{グループ~2}
\eGroupName{Group~C}

% プロジェクトリーダ
\ProjectLeader{1200000}{西谷歩}{Ayumi~Nishiya}

% グループリーダ
\GroupLeader  {1015174}{山根春貴}{Haruki~Yamane}

% メンバー数
\SumOfMembers{4}
% グループメンバ
\GroupMember  {1}{1015117}{佐藤碧}{Aoi~Sato}
\GroupMember  {2}{1015078}{佐藤礼於}{Leo~Sato}
\GroupMember  {3}{1015216}{堀沙枝香}{Saeka~Hori}
\GroupMember  {4}{1015259}{頼亜弥}{Aya~Rai}

% 指導教員
\jadvisor{伊藤恵,南部美砂子,奥野拓,原田泰}
% 複数人数いる場合はカンマ(,)で区切る。カンマの前後に空白は入れない。
\eadvisor{Kei~Ito,Misako~Nambu,Taku~Okuno,Yasushi~Harada}

% 論文提出日
\jdate{2017年7月26日}
\edate{July~26, 2017}

\begin{document}
%
% 表紙
\maketitle

%前付け
\frontmatter

% 概要
\subfile{sections/0-overview}

\tableofcontents% 目次


\mainmatter% 本文のはじまり

\chapter{背景と目的}
\subfile{sections/1-haikei}


\chapter{活動プロセス}
\subfile{sections/2-katsudou}


\chapter{知識習得}
\subfile{sections/3-tchishiki}


\chapter{システム案の変化}
\subfile{sections/4-henka}


\chapter{中間成果発表}
\subfile{sections/5-tyukan}


\chapter{学びと今後の展望}
\subfile{sections/6-kongo}


% 以降、付録(付属資料)であることを示す
\begin{appendix}

\chapter{新規習得技術}
\begin{hissu}
課題解決過程に習得した技術について解説する。
\end{hissu}

\chapter{活用した講義}
\begin{hissu}
課題解決過程において活用した講義について、講義名・活用内容を記述する。
\end{hissu}

\chapter{相互評価}
\begin{hissu}
課題解決過程で分担し、連携した作業全般について、互いに客観的に評価する。
\end{hissu}

\chapter{その他製作物}
\begin{hissu}
その他成果物をプロジェクトの担当教員の指示に従って添付する。
\end{hissu}

%付録の終わり
\end{appendix}


%\backmatter

% 参考文献
\begin{thebibliography}{9}
 \bibitem {ラベル} 著者名. 書籍名. 出版社,  年号.
 \bibitem {A2} ほげほげお. うんたらかんたら,  2003.
\end{thebibliography}

\end{document}
