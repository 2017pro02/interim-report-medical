\documentclass[../report]{subfiles}
\setcounter{section}{0}
\begin{document}

本章では、システム案出しのプロセスについて説明する。
\bunseki{山根春貴}

\section{システム案出し}
知識習得の際に得た知識からグループメンバーそれぞれが認知症に関する別々の分野の問題に着目してシステム案を考えた。
その結果、 1.排泄通知システム、 2.MyGO!、 3.介護者SNS、 4.認知症患者の見守りとプライバシーの両立、5.暴力・暴言行動記録システムの5案が出た。
アイデアの概要を以下で紹介する。
\bunseki{山根春貴}

\subsection{排泄通知システム}
\bunseki{山根春貴}

\subsection{MyGo!}
\bunseki{山根春貴}

\subsection{認知症患者の見守りとプライバシの両立}
\bunseki{山根春貴}

\subsection{暴力・暴言行動記録}
\bunseki{山根春貴}


\section{システム案レビュー}
\subsection{指導教員からのレビュー}
2017年5月26日、指導教員からシステム案出しで出たシステム案のレビューを頂いた。
頂いたレビューの一部を以下に紹介する。
\bunseki{山根春貴}

\subsection{函館認知症の人を支える会の代表者からのレビュー}
2017年5月31日、公立はこだて未来大学で行われた認知症サポーター養成講座を受講した。
この講座では、認知症患者との接し方など知識習得、自分たちで考えたシステム案に対するレビューを目的として受講した。
講師の方から頂いたレビューの一部を以下に紹介する。
\bunseki{山根春貴}

\subsection{京都府立医科大学成本医師からのレビュー}
2017年6月7日、京都府立医科大学の成本迅医師とSkypeによる会議を行った。
この会議では、システム案に対してのコメントを頂いたり、成本迅医師の要望を聞いたりした。
会議で成本迅医師から頂いた意見の一部を以下に紹介する。
\bunseki{山根春貴}

\subsection{もの忘れカフェでのレビュー}
2017年6月17日、函館総合福祉センターで行われたもの忘れカフェに参加した。
このイベントでは、施設のことや認知症患者に詳しい方にシステム案を提案し、レビューを頂いた。
時間の都合上、「排泄通知システム」と「MyGO!」の2案だけ提案した。もの忘れカフェで頂いたレビューの一部を以下に紹介する。
\bunseki{山根春貴}


\section{システム案の再検討}
\subsection{新しいシステム案}
頂いたレビューや知識習得からシステム案の改善を繰り返してきた。
しかし、どのシステム案もいくつか問題点が残るものとなり、アイデアの試行錯誤が続いた。
そこで、今までの案から建設的なアイデアが生まれることを期待して、案の再考という流れに至った。
再考するまでのシステム案は、認知症患者やその介護者に関する問題を解決するシステム案を考えていた。
再考する際、認知症になる前段階の高齢者を対象とし、システム案を考えた。
システム案を考え、グループで出た方向性を以下にまとめた。
・      認知症ではない高齢者の認知症のリスクを減らす
・      ライフログデータを活用したコミュニケーションや生活習慣の見直しの切っ掛け作りをする
この方向性を整理してまとまった案が「認知症予防のための食習慣改善システム」である。
\bunseki{山根春貴}

\subsection{新しいシステム案に対するレビュー}
\bunseki{山根春貴}

\end{document}
