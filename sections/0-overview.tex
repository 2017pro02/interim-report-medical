\documentclass[../report]{subfiles}
\begin{document}

% 和文概要
\begin{jabstract}
本プロジェクトでは、フィールドを複数設定して、それらの問題点をフィールドワークを通して発見し、ICTを活用して解決する。
それにより地域・社会に貢献することを目標として活動を行っている。
開発手法はアジャイル開発手法を用いる。
素早くアプリを開発し、それに対するレビューを受けて問題解決の質をより高いものにしていく。

現在、我が国の認知症総患者数は2012年の時点で462万人以上にも上り、年々増加傾向にある\cite{zouka}。
認知症の発症には食事、運動、睡眠などによる生活習慣病が深く関わっており、植木(2005)の調査では、健常者と比べて認知症患者の偏食が顕著であることが明らかになっている\cite{ueki}。
本グループは、高齢者の食事データを家族や医療従事者と共有することで、高齢者の食生活改善を促し、認知症予防に繋げられるのではないかと考えた。
そこで本グループは、RaspberryPiを用いて高齢者の食生活を記録し、そのデータを家族や医療従事者と共有して改善を促すシステムの開発を目的とする。

目標達成のために、本グループは高齢者でも簡単に料理を撮影することが出来るボックスと、家族や医療従事者と撮影した写真を共有できるウェブサーバを考案した。
本案では、高齢者がボックスに料理を入れることで、ボックス内のカメラが自動的に写真を撮影し、その写真をウェブサーバに送信する。
高齢者の家族や医療従事者は、このウェブサーバにアクセスすることで、高齢者が撮影した料理の写真を見ることができる。
\begin{jkeyword}
高齢者、認知症、医療、生活習慣、食習慣、ライフログ、カメラ
\end{jkeyword}
\bunseki{佐藤礼於}
\end{jabstract}

%英語の概要
\begin{eabstract}
In this project, we set up multiple fields, discover those problems through fieldwork, solve using ICT.
We are doing activities with the goal of contributing to the community and society by doing so.
Development method uses agile development method.
We will promptly develop applications and review reviews of them to make the quality of problem solving higher.

Today, the total number of patients with dementia in our country has reached more than 4,620,000 as of 2012, and it is on an increasing trend year by year.
Lifestyle-related diseases such as diet, exercise, and sleep are deeply involved in the onset of dementia, and it is clear that uneven diet of dementia patients is more noticeable in the investigation of planting trees (2005) than in healthy subjects It has become.
By thinking about sharing the dietary data of elderly people with their families and medical professionals, this group felt that the elderly would be able to improve their dietary habits and lead to the prevention of dementia.
Therefore, this group aims to develop a system that records dietary habits of elderly people using RaspberryPi, shares their data with family members and medical professionals, and encourages improvement.

In order to achieve the goal, this group devised a box that allows easy shooting of elderly people, as well as a web server that can share photos taken with families and medical professionals.
In this proposal, when the elderly puts food in the box, the camera in the box automatically takes a picture and transmits the picture to the web server.
Families and medical professionals of the elderly can see pictures of the dishes taken by the elderly by accessing this web server.
% 英文キーワード
\begin{ekeyword}
Elderly, Dementia, Medical, Lifestyle, Dietary, Lifelog, Camera
\end{ekeyword}
\bunseki{佐藤礼於}
\end{eabstract}

\end{document}
