\documentclass[../report]{subfiles}
\begin{document}

% 和文概要
\begin{jabstract}
本プロジェクトでは,3つのフィールドを調査し,そこから問題を見つけ,ICTを活用して解決する.
それにより地域・社会に貢献することを目標として活動を行っている.
開発手法はアジャイル開発手法を用いる.
素早くアプリを開発し,それに対するレビューを受けて問題解決の質をより高いものにしていく.

現在,我が国の認知症総患者数は2012年の時点で462万人に上り,今も増加の一途を辿っている.
認知症の発症には食事,運動,睡眠などによる生活習慣病が深く関わっており,健常者と比べて認知症患者の偏食が顕著であることが分かっている.
本グループは,高齢者の食事データを家族や医療従事者と共有することで,高齢者の食生活改善を促し,認知症予防に繋げられるのではないかと考えた.
それを実現するために,高齢者の食生活をICTを活用して記録・可視化できるようにする,そのデータを家族や医療従事者に提供し,高齢者自身が気付かなかった食の偏りなどを指摘できるようにすることで,食生活の改善に役立てるようなシステムを開発する.
また,そのシステムを,高齢者の中にはICTに不慣れな人々がいることを考慮し,ICTに不慣れな方でも容易に扱えるようにすること,ライフログ取得によって得られたデータを蓄積し,医療選択時に必要となる情報を残しておけるようなものにすることを目的とした.

目的達成のために,本グループは高齢者でも簡単に料理を撮影することができるボックスと,家族や医療従事者と撮影した写真を共有できるウェブサーバを考案した.
本案では,高齢者がボックスに料理を入れることで,ボックス内のカメラが自動的に写真を撮影し,その写真をウェブサーバに送信する.
高齢者自身はテレビを通して自身の食事の記録を見ることができる.
高齢者の家族や医療従事者は,このウェブサーバにアクセスすることで,高齢者が撮影した料理の写真を見ることができる.
\begin{jkeyword}
高齢者,認知症,医療,生活習慣,食習慣,ライフログ,カメラ
\end{jkeyword}
\bunseki{佐藤礼於}
\end{jabstract}

%英語の概要
\begin{eabstract}
In this project, we will investigate three fields.
We find problems from it and solve it using ICT.
We are doing activities with the goal of contributing to the community and society by doing so.
Development method uses agile development method.
We will develop applications swiftly and receive reviews on it to make the quality of problem solving higher.

Today, the total number of patients with dementia in our country has reached more than 4,620,000 as of 2012, and it is on an increasing trend year by year.
Lifestyle-related diseases such as diet, exercise, and sleep are deeply involved with the onset of dementia, and it is clear that uneven diet of dementia patients is more noticeable than healthy subjects.
This group think that the elderly people would be able to improve their dietary habits and lead to the prevention of dementia by sharing the dietary data of them with their families and medical professionals.
In order to do that, we make dietary habits of elderly people recordable and visible with ICT.
We develop the system for improving dietary habits of elderly people by having the elderly themselves point out biases of foods they did not notice and providing the data to families and healthcare workers.
Also, we aim for making the system easy to use by elderly people inexperienced in ICT and leaving required data when their families and healthcare workers choice treatments.

In order to achieve the goal, this group devised a box that allows easy taking the picture for elderly people, as well as a web server that can share pictures taken with families and medical professionals.
In this proposal, when the elderly puts food in the box, the camera in the box automatically takes a picture and transmits the picture to the web server.
The elderly can watch the record of dishes through the TV.
Families and medical professionals of the elderly can see pictures of the dishes taken by the elderly by accessing this web server.
% 英文キーワード
\begin{ekeyword}
Elderly, Dementia, Medical, Lifestyle, Dietary, Lifelog, Camera
\end{ekeyword}
\bunseki{佐藤礼於}
\end{eabstract}

\end{document}
