\documentclass[../report]{subfiles}
\begin{document}

% 和文概要
\begin{jabstract}
本プロジェクトでは、フィールドを複数設定して、それらの問題点をフィールドワークを通して発見し、ICTを活用して解決する。
それにより地域・社会に貢献することを目標として活動を行っている。
開発手法はアジャイル開発手法を用いる。
素早くアプリを開発し、それに対するレビューを受けて問題解決の質をより高いものにしていく。

現在、我が国の認知症総患者数は2011年の時点で500万人以上にも上り、年々増加傾向にある。
認知症の発症には食事、運動、睡眠などによる生活習慣病が深く関わっており、植木(2005)の調査では、健常者と比べて認知症患者の偏食が顕著であることが明らかになっている。
本グループは、高齢者の食事データを家族や医療従事者と共有することで、高齢者の食生活改善を促し、認知症予防に繋げられるのではないかと考えた。
そこで本グループは、RaspberryPiを用いて高齢者の食生活を記録し、そのデータを家族や医療従事者と共有して改善を促すシステムの開発を目的とする。

目標達成のために、本グループは高齢者でも簡単に料理を撮影することが出来るボックスと、家族や医療従事者と撮影した写真を共有できるウェブサーバを考案した。
本案では、高齢者がボックスに料理を入れることで、ボックス内のカメラが自動的に写真を撮影し、その写真をウェブサーバに送信する。
高齢者の家族や医療従事者は、このウェブサーバにアクセスすることで、高齢者が撮影した料理の写真を見ることができる。
\begin{jkeyword}
高齢者、認知症、医療、生活習慣、食習慣、ライフログ、カメラ
\end{jkeyword}
\bunseki{佐藤礼於}
\end{jabstract}

%英語の概要
\begin{eabstract} Abstract in English.
\fake{you should write your English abstract in one page. }
% 英文キーワード
\begin{ekeyword}
Keyrods1, Keyword2, Keyword3, Keyword4, Keyword5
\end{ekeyword}
\bunseki{佐藤礼於}
\end{eabstract}

\end{document}
