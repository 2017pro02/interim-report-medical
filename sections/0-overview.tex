\documentclass[../report]{subfiles}
\begin{document}

本プロジェクトでは、フィールドを複数設定して、それらの問題点をフィールドワークを通して見つけ、ICTを活用して解決する。
それにより地域・社会に貢献することを目標として活動を行っている。
開発手法はアジャイル開発手法を用いる。
素早くアプリを開発し、それに対するレビューを受けて問題解決の質をより高いものにしていく。

現在、認知症患者は2011年の時点で日本に500万人以上いて、増加傾向にある。
認知症の発症には生活習慣病が深く関わっており、植木(2005)の調査では、健常者と比べて認知症患者の偏食が顕著であることが分かっている。
本グループは、高齢者の食事をRaspberryPiで自動で撮影してサーバに送信し、家族や医療従事者に提供することで、食生活改善を促し、ひいては認知症予防に繋げられるのではないかと考えた。
そこで本グループは、RaspberryPiで高齢者の食生活を記録し、家族や医療従事者に提供して改善を促すシステムの開発を目的とする。

目標達成のために、本グループは高齢者でも気軽に食事のログを取ることが出来るボックスと、家族や医療従事者と食事の写真を共有できるウェブサーバを考案した。
本案では、高齢者がボックスに食事を入れることで、ボックス内のカメラが自動的に写真を撮影し、その写真をウェブサーバに送信する。
高齢者の家族や医療従事者は、このウェブサーバにアクセスすることで、高齢者の摂った食事を見ることができる。

\end{document}
