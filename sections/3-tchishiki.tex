\documentclass[../report]{subfiles}
\setcounter{section}{0}
\begin{document}

本章では,認知症に関する知識習得について説明する.
\bunseki{佐藤碧}

\section{書籍・論文}
プロジェクト開始時点ではグループメンバーは,認知症についての知識を殆ど持っていない状態であった.
そこでまずはそもそも認知症とは何なのかや,その症状や療法,認知症に関する現状の社会問題や取り組みについて,書籍やインターネット上に公開されている論文等を個人で調べ,認知症に関する基本的な知識を身につけた.
それぞれが得た知識に関しては,スライドやドキュメントにまとめグループメンバー全員で共有することで,認知症に関する様々な分野について多角的に知ることができ,その結果,グループ全体として効率よく知識習得を進めることができた.
メンバーが読んだ本については,参考文献に示した.
\bunseki{佐藤碧}

\section{認知症サポーター養成講座}
函館認知症の人を支える会の方を講師として大学にお招きし,認知症とは何かや,認知症を引き起こす原因,中核症状と行動・心理症状について,認知症の人の支援の仕方や出会った際の接し方などといった内容の講座を受けた.
それまでの文献からだけの知識ではなく,実際に認知症の方と接している人たちならではの,切実な思いや介護負担の大きさについて知ることができた.
特に,論文等ではどうしても研究目的で書かれてしまうものであるが,そうではなく,実際にその研究の対象になっている方や利用することになるであろう方たちの思いや考え方に触れることができたことは,認知症の問題に取り組んでいく上での貴重な機会となった.
\bunseki{佐藤碧}

\section{ロボット開発ワークショップ}
京都の同志社女子大学京田辺キャンパスで開催された,ロボット開発ワークショップに参加した.
このワークショップでは,自身の価値にまつわる過去の経験を説明し,他者とのコミュニケーションの中で互いの価値や本質を見つけ出すためのシナリオ作成をした.
それを通じて,コミュニケーションロボットが人間とのコミュニケーションを行う上で必要となることは何なのかということを見つけ出そうという試みであった.
認知症との直接的な関係はあるとは言えなかったが,認知症治療における意思決定支援のシステムを構築する上では重要になる要素であると考えられる.
認知症治療や介護においてはまだまた人の手で行わなければならないことが多いが,それは認知症のもつ中核症状が実に多様であり,それを引き起こす原因がその人の周囲の環境によるものであることが理由として考えられる.
そのように多様性を示すものに関してはまだまだコンピュータが対処できない問題であるが,それらを解決していく上で,このように人の中心となるような価値や本質を知るという極めて知的な行いを解明しようという試みの一端に触れることができたのは貴重な経験であったと言える.
\bunseki{佐藤碧}

\end{document}
