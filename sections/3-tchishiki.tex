\documentclass[../report]{subfiles}
\setcounter{section}{0}
\begin{document}

本章では,認知症に関する知識習得について説明する.
\bunseki{佐藤碧}

\section{書籍・論文}
プロジェクト開始時点で本グループメンバーは,認知症についての知識を殆ど持っていない状態であった.
そこでまずはそもそも認知症とは何か,その症状や療法,認知症に関する現状の社会問題や取り組みについて,書籍やインターネット上に公開されている論文等を個人で調べ,認知症に関する基本的な知識を身につけた.
それぞれが得た知識に関しては,スライドやドキュメントにまとめメンバー全員と共有した.
その結果,認知症に関する様々な分野について多角的に知ることができ,グループ全体として効率良く知識習得を進めることができた.
メンバーが読んだ本については,参考文献の[7]から[14]に示した.
\bunseki{佐藤碧}

\section{認知症サポーター養成講座}
函館認知症の人を支える会の方を講師として大学にお招きした.
認知症とは何か,認知症を引き起こす原因,中核症状と行動・心理症状について,また認知症の人の支援の仕方や出会った際の接し方などといった内容の講座を受けた.
それまでの文献から得た知識だけではなく,実際に認知症の人と接している方々ならではの切実な思いや,介護負担の大きさについて知ることができた.
実際にその研究の対象になっている方や関係者の方々の思いや考え方に触れることができたことは,認知症の問題に取り組んでいく上での貴重な機会となった.
\bunseki{佐藤碧}

\section{ロボット開発ワークショップ}
京都の同志社女子大学京田辺キャンパスで開催された,ロボット開発ワークショップに参加した.
このワークショップでは,自身の価値にまつわる過去の経験を説明し,他者とのコミュニケーションの中で互いの価値や本質を見つけ出すためのシナリオ作成をした.
それを通じて,コミュニケーションロボットが人間とのコミュニケーションを行う上で必要となることを見つけ出そうという試みであった.
京都府立医科大学の成本医師から本人の生活パターンや好みを基に認知症患者本人の意思を判断していることがあるというご意見を頂いていたので,他者の価値や本質を見つけ出す試みは認知症治療における意思決定支援のシステムを構築する上で重要になる要素であると考えられる.

認知症治療や介護において,コンピュータが発達してきた現代でもまだまだ人の手で行わなければならないことが多い.
それは認知症の人の周囲環境や家族との人間関係など,未だにコンピュータでは簡単にパターン化できないこともあるからである.
今回のワークショップでは,それらを解決していく上で重要になってくる人の中核部分となるような価値や本質を知るという試みであったため,その体験ができたことは貴重な経験であったと言える.
\bunseki{佐藤碧}

\end{document}
