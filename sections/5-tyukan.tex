\documentclass[../report]{subfiles}
\setcounter{section}{0}
\begin{document}

本章では,中間発表について説明する.

\section{学び}
本グループは7月14日の中間発表会で,「認知症予防のための生活習慣改善システム」について発表した.発表形式について,ポスターセッションを用いてシステム案について解説した.

本グループの発表に対し,頂いた評価シートの枚数は14枚であった.

発表技術の評価について,平均点は6.5であった.声が聞きやすいというご意見のほかに,ポスターだけでなく,スライドを用いて発表したほうが分かりやすいというご意見を複数頂いた.

また,発表内容の評価について,平均点は6.7であった.システム案に対して「自身が閲覧」のときにどういう食事の仕方がいいかなど基準となるデータがあればいいと思う.」,「動作をデバイスを使用するときも簡単にはなっているが,「置く」という動作が必要になってしまう.対象者には動作なく自動で撮影できているのがベストではあるからそこをうまく考えられたら.」などの具体的なご意見を頂いた.
\bunseki{頼亜弥}

\section{自己評価}
本グループは評価内容を踏まえて,目的,現状の把握,今後の計画の具体性,表現力,チームワークをもとに本グループの発表を5段階で客観的に評価し,その根拠について以下に示す.

\begin{itemize}
    \item 目的:5
    
中間発表を通して,本グループのシステムに関する具体的なご意見を頂き,今後の課題がより明確なものとなった.
    \item 現状の把握:4
   
 本グループは今までの活動についてポスターにまとめており,ポスター自体に対して「わかりやすい」という評価が多数あった.また,本グループの発表に対して,「質疑応答含めて聞いてる人とコミュニケーションが取れていてよかった」というご意見も頂いた.
    \item 今後の計画の具体性:3
  
  今後の計画について,システム開発はまだ案を詰めている段階にあり,今後他の生活習慣に対するアプローチをどうしていくかということも含め,具体的な計画をしっかり行っていきたいと考えている.
    \item 表現力:3

    中間発表で頂いた評価を見ると,発表に対する評価にばらつきがあった.分かりやすい説明であるという評価を多数得たが,声が小さいというご意見も頂いた.
    \item チームワーク:5

    本グループの発表では,発表者以外のグループメンバーが評価シートを配り,しっかり発表者のサポートをしていた.
\end{itemize}
\bunseki{頼亜弥}

\section{今後の課題}
本グループは,中間発表会で頂いたご意見を受け止め,今後のプロジェクト活動の課題としてメンバー全員で見直したいと考えている.今後の課題を具体的に以下に示す.
発表技術について,最終発表会ではポスターセッションのほかにスライドを用いることで,よりわかりやすい発表を目指していきたいと考えている.

発表内容について,「高齢者への撮影した写真データの見せ方」,「料理の撮影方法」などのシステム内容の改善を課題として,システム開発に取り組んでいきたいと考えている.
\bunseki{頼亜弥}

\end{document}
